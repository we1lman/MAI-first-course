\documentclass{article}
\author{Andrey Velma}
\usepackage[russian]{babel}
\usepackage[utf8]{inputenc}
\usepackage{geometry}
\usepackage[12pt]{extsizes}
\usepackage{setspace}
\usepackage{amsmath}
\usepackage{tikz}  
\usepackage{soul}
\usepackage{xcolor}
\usepackage{circledsteps}
\usepackage{colortbl}
\usetikzlibrary{shapes.geometric}
\usepackage{verbatim}
\usetikzlibrary{graphs}
\usetikzlibrary{tikzmark,overlay-beamer-styles}
\usetikzlibrary{arrows}
\geometry{
	a4paper,
	top=3mm, 
	right=4mm, 
	%bottom=4mm, 
	left=4mm,
	head=-1mm
}
\tikzstyle{materia}=[draw, text width=6.0em, text centered,
minimum height=1.5em]
\tikzstyle{block} = [materia, text width=8em, minimum width=10em,
minimum height=3em, rounded corners]

 


\begin{document}
\voffset = 8mm
\begin{flushleft}
	\begin{Large}
		\\
		\hspace{8mm}\textbf{Вариант 4}
	\end{Large}
\end{flushleft}
\vspace{8mm}
\large
\hspace{8mm}\textbf{1.} Определить для орграфа, заданного матрицей смежности:
\vspace{5mm}
\\
\hspace*{50mm}
$A =
	\begin{pmatrix}
		0 & 1 & 1 & 0 \\
		1 & 0 & 1 & 0 \\
		0 & 0 & 0 & 0 \\
		1 & 1 & 0 & 0 \\
	\end{pmatrix}$
\vspace{5mm}
\\
\hspace*{12mm} а) матрицу односторонней связности(2 способа, включая итерационный алгоритм);
\\
\hspace*{12mm} б) матрицу сильной связности;
\\
\hspace*{12mm} в) компоненты сильной связности;
\\
\hspace*{12mm} г) матрицу контуров;
\\
\hspace*{12mm} д) изображение графа и компонент сильной связности.
\\
\\
\hspace*{8mm}\textbf{2.} Используя алгоритм Терри, определить замкнутый маршрут, проходящий ровно
\\
\hspace*{12mm} по два раза (по одному в каждом направлении) через каждое ребро графа.
\vspace{5mm}
\\
\begin{center}
    \begin{tikzpicture}
        [node distance={25mm}, thick, main/.style = {draw, circle}]
        \node[main] (2) at (-2, 3) {$V_2$};
        \node[main] (3) at (4, 3) {$V_3$};
        \node[main] (4) at (1, 1.5) {$V_4$};
        \node[main] (1) at (-2, 0) {$V_1$};
        \node[main] (5) at (4, 0) {$V_5$};
        \draw (1) -- (2);
        \draw (1) -- (4);
        \draw (1) -- (5);
        \draw (2) -- (3);
        \draw (2) -- (4);
        \draw (3) -- (4);
        \draw (3) -- (5);
    \end{tikzpicture}
\end{center}
\\
\\
\vspace{10mm}
\hspace*{8mm}\textbf{3.} Используя алгоритм “фронта волны”, найти все минимальные пути из первой
\\
\hspace*{12mm} вершины в последнюю орграфа, заданного матрицей смежности.
\vspace{5mm}
\\
\\

\hspace*{40mm}
$A =
	\begin{pmatrix}
		0 & 0 & 0 & 1 & 1 & 0 & 0 \\
		1 & 0 & 1 & 1 & 0 & 1 & 1 \\
		1 & 1 & 0 & 1 & 1 & 0 & 1 \\
		0 & 0 & 0 & 0 & 1 & 1 & 0 \\
		1 & 0 & 0 & 1 & 0 & 1 & 0 \\
		0 & 1 & 1 & 1 & 0 & 0 & 0 \\
		1 & 1 & 0 & 0 & 1 & 1 & 0 \\
	\end{pmatrix}$

\newpage
\hoffset=2mm
\textbf{4.} Используя алгоритм Форда, найти минимальные пути из первой вершины во все
\\ \hspace*{10mm} достижимые вершины в нагруженном графе, заданном матрицей длин дуг.
\vspace{5mm}
\\
\hspace*{40mm}
$C =
	\begin{pmatrix}
		\infty & 3      & 5      & \infty & 6        & \infty & \infty & \infty \\
		2      & \infty & 1      & 4      & \infty   & \infty & \infty & \infty \\
		3      & \infty      & \infty & 4      & 2 & \infty      & \infty & \infty \\
		\infty & \infty & \infty      & \infty & \infty & 3 & 5      & \infty \\
		4      & \infty & \infty & \infty & \infty & 6      & \infty & 7      \\
		\infty & \infty & \infty & \infty & \infty      & \infty & 3      & 2      \\
		6      & \infty & \infty & \infty & \infty & \infty      & \infty & 1      \\
		8      & \infty & \infty & \infty & 11     & \infty & \infty & \infty \\
	\end{pmatrix}$
\vspace{5mm}
\\
\hspace*{4mm} \textbf{5.}  Найти остовное дерево с минимальной суммой длин входящих в него ребер. Значения X1 - X13 приведены в задании, значения X14 - X17 равны 5.

\\
\begin{center}
	\begin{tikzpicture}
		[node distance={20mm}, thick, main/.style = {draw, circle, fill=black}]
		\node[main] (1) {};
		\node[main] (2) [right of = 1] {};
		\node[main] (3) [right of = 2]{};
		\node[main] (4) [right of = 3]{};
		\node[main] (5) [below of = 1]{};
		\node[main] (6) [below of = 2]{};
		\node[main] (7) [below of = 3] {};
		\node[main] (8) [below of = 4] {};
		\node[main] (9) [below of = 5] {};
		\node[main] (10) [below of = 6] {};
		\node[main] (11) [below of = 7] {};
		\node[main] (12) [below of = 8] {};
		\draw (1) -- node[above] {2} (2);
		\draw (2) -- node[above] {6} (3);
		\draw (3) -- node[above] {7} (4);
		\draw (1) -- node[left] {1} (5);
		\draw (2) -- node[left] {5} (6);
		\draw (3) -- node[left] {5} (7);
		\draw (4) -- node[right] {6} (8);
		\draw (5) -- node[above] {8} (6);
		\draw (6) -- node[above] {4} (7);
		\draw (7) -- node[above] {1} (8);
		\draw (5) -- node[left] {7} (9);
		\draw (6) -- node[left] {5} (10);
		\draw (7) -- node[left] {5} (11);
		\draw (8) -- node[right] {5} (12);
		\draw (9) -- node[below] {9} (10);
		\draw (10) -- node[below] {7} (11);
		\draw (11) -- node[below] {3} (12);
	\end{tikzpicture}
\end{center}\\

\hspace*{4mm} \textbf{6.}  Пусть каждому ребру неориентированного графа соответствует некоторый
\\ \hspace*{12mm}элемент электрической цепи. Составить линейно независимые системы
\\ \hspace*{12mm}уравнений Кирхгофа для токов и напряжений. Пусть первому и пятому ребру
\\ \hspace*{12mm}соответствуют источники тока с ЭДС $E1$ и $E2$, а остальные элементы являются
\\ \hspace*{10mm} сопротивлениями.  Используя закон Ома, и, предполагая внутренние
\\ \hspace*{12mm}сопротивления источников тока равными нулю, получить систему уравнений для
\\ \hspace*{10mm} токов.
\begin{center}
    \begin{tikzpicture}
        [node distance={30mm}, thick, main/.style = {draw, circle}]
        \node[main] (1) at (-3, 0) {$1$};
        \node[main] (2) at (1, 2) {$2$};
        \node[main] (3) at (5, 0) {$3$};
        \node[main] (4) at (4, -3) {$4$};
        \node[main] (5) at (0, -3) {$5$};
        
        \draw (1) -- node[above left] {1} (2);
        \draw (1) -- node[above] {6} (3);
        \draw (1) -- node[above right] {7} (4);
        \draw (1) -- node[left] {5} (5);
        \draw (2) -- node[above right] {2} (3);
        \draw (2) -- node[right] {8} (4);
        \draw (3) -- node[right] {3} (4);
        \draw (3) -- node[below right] {9} (5);
        \draw (4) -- node[below] {4} (5);
    \end{tikzpicture}
\end{center}
\newpage
\noindent
\hspace*{4mm} \textbf{7.} Построить максимальный поток по транспортной сети.
\vspace{5mm}
\\
\begin{tikzpicture}
	[node distance={20mm}, thick, main/.style = {draw, circle}]
	\hspace*{30mm}
	\node[main] (1) {$1$};
	\node[main] (2) [above right of = 1, above = 20pt, right = 3pt] {$2$};
	\node[main] (3) [above right of = 2, above = -5mm, right = 40pt, node distance={20mm}]{$3$};
	\node[main] (4) [below right of = 3, below = -5mm, right = 40pt]{$7$};
	\node[main] (5) [right of = 1, node distance={45mm}]{$5$};
	\node[main] (6) [right of = 5, node distance={65mm}]{$9$};
	\node[main] (7) [below right of = 1, below = 20pt, right = 3pt] {$4$};
	\node[main] (8) [below right of = 7, below = -5mm, right = 40pt, node distance={20mm}]{$6$};
	\node[main] (9) [above right of = 8, above= -5mm, right = 40pt]{$8$};
	\draw[->] (1) -- node[above] {3} (2);
	\draw[->] (2) -- node[above] {3} (3);
	\draw[->] (3) -- node[above] {9} (4);
	\draw[->] (4) -- node[above] {11} (6);
	\draw[->] (1) to [out=0,in=230,looseness=0.5] node[above = -2mm, left = 5mm] {10} (3);
	\draw[->] (1) -- node[above = 3mm, right=5mm] {5} (5);
	\draw[->] (5) -- node[above] {4} (6);
	\draw[->] (1) -- node[below] {3} (7);
	\draw[->] (7) -- node[below] {3} (8);
	\draw[->] (8) -- node[below] {7} (9);
	\draw[->] (9) -- node[below] {13} (6);
	\draw[->] (1) to [out=0,in=130,looseness=0.5] node[below = -2mm, left = 5mm] {7} (8);
	\draw[->] (2) -- node[above = 6mm, left = 1mm] {2} (5);
	\draw[->] (5) -- node[above] {3} (4);
	\draw[->] (7) -- node[above] {4} (5);
	\draw[->] (5) -- node[above] {5} (9);
\end{tikzpicture}
\vspace{5mm}
\\

\newpage
\large
\begin{center}
	\textbf{Задание №1}
\end{center}
\par
а)
\vspace{5mm}
\par \hspace{8mm} $A =
	\begin{pmatrix}
		0 & 1 & 1 & 0 \\
		1 & 0 & 1 & 0 \\
		0 & 0 & 0 & 0 \\
		1 & 1 & 0 & 0 \\
	\end{pmatrix}$
\vspace{5mm}
\par
\hspace{8mm} $A^2 = A * A =
	\begin{pmatrix}
		0 & 1 & 1 & 0 \\
		1 & 0 & 1 & 0 \\
		0 & 0 & 0 & 0 \\
		1 & 1 & 0 & 0 \\
	\end{pmatrix}$
$*\begin{pmatrix}
		0 & 1 & 1 & 0 \\
		1 & 0 & 1 & 0 \\
		0 & 0 & 0 & 0 \\
		1 & 1 & 0 & 0 \\
	\end{pmatrix}$ =
$\begin{pmatrix}
		1 & 0 & 1 & 0 \\
		0 & 1 & 1 & 0 \\
		0 & 0 & 0 & 0 \\
		1 & 1 & 1 & 0 \\
	\end{pmatrix}$
\vspace{5mm}
\par
\hspace{8mm} $A^3 = A^2 * A =
	\begin{pmatrix}
		1 & 0 & 1 & 0 \\
		0 & 1 & 1 & 0 \\
		0 & 0 & 0 & 0 \\
		1 & 1 & 1 & 0 \\
	\end{pmatrix}$
$*\begin{pmatrix}
		0 & 1 & 1 & 0 \\
		1 & 0 & 1 & 0 \\
		0 & 0 & 0 & 0 \\
		1 & 1 & 0 & 0 \\
	\end{pmatrix}$ =
$\begin{pmatrix}
		0 & 1 & 1 & 0 \\
		1 & 0 & 1 & 0 \\
		0 & 0 & 0 & 0 \\
		1 & 1 & 1 & 0 \\
	\end{pmatrix}$
\vspace{5mm}
\par
\hspace{8mm}
$T = E \vee A \vee A^2 \vee A^3 =
	\begin{pmatrix}
		1 & 1 & 1 & 0 \\
		1 & 1 & 1 & 0 \\
		0 & 0 & 1 & 0 \\
		1 & 1 & 1 & 1 \\
	\end{pmatrix}$
\vspace{5mm}
\par
\hspace{8mm}
\hspace*{-15mm} \centerline{По итерационному алгоритму Уоршалла.}
\hspace*{-15mm} \centerline{k = 0}
\vspace{5mm}
\par
\hspace{8mm} $T^0 = E ∪ A =
	\begin{pmatrix}
		1 & 0 & 0 & 0 \\
		0 & 1 & 0 & 0 \\
		0 & 0 & 1 & 0 \\
		0 & 0 & 0 & 1 \\
	\end{pmatrix}$
$∪\begin{pmatrix}
		0 & 1 & 1 & 0 \\
		1 & 0 & 1 & 0 \\
		0 & 0 & 0 & 0 \\
		1 & 1 & 0 & 0 \\
	\end{pmatrix}$ =
$\begin{pmatrix}
		1 & 1 & 1 & 0 \\
		1 & 1 & 1 & 0 \\
		0 & 0 & 1 & 0 \\
		1 & 1 & 0 & 1 \\
	\end{pmatrix}$
\vspace{5mm}
\par
\hspace{8mm}
\hspace*{55mm}{k = 1, k - 1 = 0}
\vspace{5mm}
\par
\hspace{8mm} $T^1 =
	\begin{pmatrix}
		1 & 1 & 1 & 0 \\
		1 & 1 & 1 & 0 \\
		0 & 0 & 1 & 0 \\
		1 & 1 & 0 & 1 \\
	\end{pmatrix}$
$∪\begin{pmatrix}
		1 & 1 & 1 & 0 \\
		1 & 1 & 1 & 0 \\
		0 & 0 & 0 & 0 \\
		1 & 1 & 1 & 0 \\
	\end{pmatrix}$ =
$\begin{pmatrix}
		1 & 1 & 1 & 0 \\
		1 & 1 & 1 & 0 \\
		0 & 0 & 1 & 0 \\
		1 & 1 & 1 & 1 \\
	\end{pmatrix}$
\vspace{5mm}
\par
\hspace{8mm}
\hspace*{55mm}{k = 2, k - 1 = 1}
\vspace{5mm}
\par
\hspace{8mm} $T^2 =
	\begin{pmatrix}
		1 & 1 & 1 & 0 \\
		1 & 1 & 1 & 0 \\
		0 & 0 & 1 & 0 \\
		1 & 1 & 1 & 1 \\
	\end{pmatrix}$
$∪\begin{pmatrix}
		1 & 1 & 1 & 0 \\
		1 & 1 & 1 & 0 \\
		0 & 0 & 0 & 0 \\
		1 & 1 & 1 & 0 \\
	\end{pmatrix}$ =
$\begin{pmatrix}
		1 & 1 & 1 & 0 \\
		1 & 1 & 1 & 0 \\
		0 & 0 & 1 & 0 \\
		1 & 1 & 1 & 1 \\
	\end{pmatrix}$
\vspace{5mm}
\par
\hspace{8mm}
\hspace*{55mm}{k = 3, k - 1 = 2}
\vspace{5mm}
\par
\hspace{8mm} $T^3 =
	\begin{pmatrix}
		1 & 1 & 1 & 0 \\
		1 & 1 & 1 & 0 \\
		0 & 0 & 1 & 0 \\
		1 & 1 & 1 & 1 \\
	\end{pmatrix}$
$∪\begin{pmatrix}
		0 & 0 & 1 & 0 \\
		0 & 0 & 1 & 0 \\
		0 & 0 & 1 & 0 \\
		0 & 0 & 1 & 0 \\
	\end{pmatrix}$ =
$\begin{pmatrix}
		1 & 1 & 1 & 0 \\
		1 & 1 & 1 & 0 \\
		0 & 0 & 1 & 0 \\
		1 & 1 & 1 & 1 \\
	\end{pmatrix}$
\vspace{5mm}
\par
\hspace{8mm}
\hspace*{55mm}{k = 4, k - 1 = 3}
\vspace{5mm}
\par
\hspace{8mm} $T^4 =
	\begin{pmatrix}
		1 & 1 & 1 & 0 \\
		1 & 1 & 1 & 0 \\
		0 & 0 & 1 & 0 \\
		1 & 1 & 1 & 1 \\
	\end{pmatrix}$
$∪\begin{pmatrix}
		0 & 0 & 0 & 0 \\
		0 & 0 & 0 & 0 \\
		0 & 0 & 0 & 0 \\
		1 & 1 & 1 & 1 \\
	\end{pmatrix}$ =
$\begin{pmatrix}
		1 & 1 & 1 & 0 \\
		1 & 1 & 1 & 0 \\
		0 & 0 & 1 & 0 \\
		1 & 1 & 1 & 1 \\
	\end{pmatrix} = T$
\vspace{5mm}
\\

б) $\overline{S} = T \& T^T = \begin{pmatrix}
		1 & 1 & 1 & 0 \\
		1 & 1 & 1 & 0 \\
		0 & 0 & 1 & 0 \\
		1 & 1 & 1 & 1 \\
	\end{pmatrix} \&
	\begin{pmatrix}
		1 & 1 & 0 & 1 \\
		1 & 1 & 0 & 1 \\
		1 & 1 & 1 & 1 \\
		0 & 0 & 0 & 1 \\
	\end{pmatrix} =
	\begin{pmatrix}
		1 & 1 & 0 & 0 \\
		1 & 1 & 0 & 0 \\
		0 & 0 & 1 & 0 \\
		0 & 0 & 0 & 1 \\
	\end{pmatrix}$
\vspace{5mm}
\par
\hspace{5mm}$\overline{S} = \begin{pmatrix}
		1 & 1 & 0 & 0 \\
		1 & 1 & 0 & 0 \\
		0 & 0 & 1 & 0 \\
		0 & 0 & 0 & 1 \\
	\end{pmatrix}$ - матрица сильной связности
\vspace{5mm}
\par


в)
\begin{center}
    \hspace{6mm}Первая компонента сильной сзязности:
    \{{$V_1}, {$V_2}\}
\end{center}
\\

\begin{center}
    \hspace{6mm}Вторая компонента сильной сзязности:
    \{{$V_3}\}
\end{center}
\\

\begin{center}
    \hspace{6mm}Третья компонента сильной сзязности:
    \{{$V_4}\}
\end{center}


\newpage
г) $ K = \overline{S} \& A = \begin{pmatrix}
		1 & 1 & 0 & 0 \\
		1 & 1 & 0 & 0 \\
		0 & 0 & 1 & 0 \\
		0 & 0 & 0 & 1 \\
	\end{pmatrix} \&
	\begin{pmatrix}
		0 & 1 & 1 & 0 \\
		1 & 0 & 1 & 0 \\
		0 & 0 & 0 & 0 \\
		1 & 1 & 0 & 0 \\
	\end{pmatrix} =
	\begin{pmatrix}
		0 & 1 & 0 & 0 \\
		1 & 0 & 0 & 0 \\
		0 & 0 & 0 & 0 \\
		0 & 0 & 0 & 0 \\
	\end{pmatrix}$
\begin{center}
	\hspace{6mm}Следовательно дуги <{$V_1$}, {$V_2$}>
	  <{$V_2$}, {$V_1$}>
	\hspace{6mm}
	принадлежат какому-либо контуру исходного графа.
\end{center}
\vspace{5mm}
\par
д) Компонентная связанность
\begin{center}
	\begin{tikzpicture}
		[node distance={40mm}, thick, main/.style = {draw, circle}]
		\hspace*{-25mm}

		\node[main, draw=red] (1) {$1$};
		\node[main, draw=red] (2) [right of = 1] {$2$};
		\node[main, draw=blue] (4) [below of= 2] {$4$};
		\node[main,draw=green] (3) [below of= 1] {$3$};
		\draw[->, >=stealth] (1) -- node[above] {} (2);
		\draw[<->, >=stealth] (1) -- node[above] {} (3);
		\draw[->, >=stealth] (1) -- node[above] {} (4);
		\draw[->, >=stealth] (3) -- node[above] {} (2);
		\draw[->, >=stealth] (3) -- node[above] {} (4);
		\draw[<->, >=stealth] (2) -- node[above] {} (4);


	\end{tikzpicture}
\end{center}
\hspace{6mm}{V_1} = \{{v_1}, {v_2}\};
{V_2} = \{{v_3}\};
{V_3} = \{{v_4}\}
\newpage
\\

\begin{center}
	\textbf{Задание №2}
\end{center}
    
\begin{center}
    \begin{tikzpicture}
        [node distance={25mm}, thick, main/.style = {draw, circle}]
        \node[main] (2) at (-2, 3) {$V_2$};
        \node[main] (3) at (4, 3) {$V_3$};
        \node[main] (4) at (1, 1.5) {$V_4$};
        \node[main] (1) at (-2, 0) {$V_1$};
        \node[main] (5) at (4, 0) {$V_5$};
        
        \draw (1) -- node[right] {1 $\uparrow$} node[left] {$\downarrow$ 14} (2);
        \draw (1) -- node[below, rotate=30] {$\leftarrow$ 10} node[above, rotate=30] {5 $\rightarrow$} (4);
        \draw (1) -- node[below, rotate=0] {11 $\rightarrow$} node[above, rotate=0] {$\leftarrow$ 4} (5);
        \draw (2) -- node[above] {2 $\rightarrow$} node[below] {$\leftarrow$ 13} (3);
        \draw (2) -- node[below, rotate=-30] {$\leftarrow$ 8} node[above, rotate=-30] {9 $\rightarrow$} (4);
        \draw (3) -- node[above, rotate=30] {6 $\rightarrow$} node[below, rotate=30] {$\leftarrow$ 7} (4);
        \draw (3) -- node[right] {12 $\uparrow$} node[left] {$\downarrow$ 3} (5);
    \end{tikzpicture}
\end{center}
\\

\begin{center}
	1 $\rightarrow$ 2 $\rightarrow$  3 $\rightarrow$  5 $\rightarrow$ 1  $\rightarrow$  4 $\rightarrow$  3 $\rightarrow$  4 $\rightarrow$  2 $\rightarrow$  4 $\rightarrow$  1 $\rightarrow$  5 $\rightarrow$ 3 $\rightarrow$ 2 $\rightarrow$ 1
\end{center}
\par
\newpage
\begin{center}
	\textbf{Задание №3}

	\vspace{30}
	\begin{pmatrix}
		0 & 0 & 0 & 1 & 1 & 0 & 0 \\
		1 & 0 & 1 & 1 & 0 & 1 & 1 \\
		1 & 1 & 0 & 1 & 1 & 0 & 1 \\
		0 & 0 & 0 & 0 & 1 & 1 & 0 \\
		1 & 0 & 0 & 1 & 0 & 1 & 0 \\
		0 & 1 & 1 & 1 & 0 & 0 & 0 \\
		1 & 1 & 0 & 0 & 1 & 1 & 0 \\
	\end{pmatrix}
\end{center}

\begin{center}
	\\
\end{center}

\vspace{2mm}
\begin{center}
    \begin{tikzpicture}
        [node distance={30mm}, thick, main/.style = {draw, circle}]
        
        \node[main] (1) at (0, 3) {$V_1$};
        \node[main] (4) at (-2, 1.5) {$V_4$};
        \node[main] (5) at (2, 1.5) {$V_5$};
        \node[main] (6) at (0, 0) {$V_6$};
        \node[main] (2) at (-2, -1.5) {$V_2$};
        \node[main] (3) at (2, -1.5) {$V_3$};
        \node[main] (7) at (0, -3) {$V_7$};

        \draw[->, >=stealth] (1) -- (4);
        \draw[->, >=stealth] (1) -- (5);
        \draw[->, >=stealth] (4) -- (6);
        \draw[->, >=stealth] (5) -- (6);
        \draw[->, >=stealth] (6) -- (2);
        \draw[->, >=stealth] (6) -- (3);
        \draw[->, >=stealth] (2) -- (7);
        \draw[->, >=stealth] (3) -- (7);
        
        \node[right of=1, node distance=31mm] {$--- W_0 (V_1)--- 0$};
        \node[right of=5, node distance=35mm] {$--- W_1 (V_4,V_5)--- 1$};
        \node[right of=6, node distance=31mm] {$--- W_2 (V_6)--- 2$};
        \node[right of=3, node distance=35mm] {$--- W_3 (V_2,V_3)--- 3$};
        \node[right of=7, node distance=31mm] {$--- W_4 (V_7)--- 4$};
        
    \end{tikzpicture}
\end{center}

\begin{center}
	\\
\end{center}

\begin{center}
	\hspace{6mm}Минимальные пути из $V_1$ в $V_7$: \\
    1. $V_1$ - $V_4$ - $V_6$ - $V_2$ - $V_7$ \\
    2. $V_1$ - $V_4$ - $V_6$ - $V_3$ - $V_7$ \\
    3. $V_1$ - $V_5$ - $V_6$ - $V_2$ - $V_7$ \\
    4. $V_1$ - $V_5$ - $V_6$ - $V_3$ - $V_7$ \\
\end{center}

\newpage
\begin{center}
	\textbf{Задание №4}
\end{center}
\hspace{10mm}
\begin{center}
    $C = 
    \begin{pmatrix}
		\infty & 3      & 5      & \infty & 6        & \infty & \infty & \infty \\
		2      & \infty & 1      & 4      & \infty   & \infty & \infty & \infty \\
		3      & \infty      & \infty & 4      & 2 & \infty      & \infty & \infty \\
		\infty & \infty & \infty      & \infty & \infty & 3 & 5      & \infty \\
		4      & \infty & \infty & \infty & \infty & 6      & \infty & 7      \\
		\infty & \infty & \infty & \infty & \infty      & \infty & 3      & 2      \\
		6      & \infty & \infty & \infty & \infty & \infty      & \infty & 1      \\
		8      & \infty & \infty & \infty & 11     & \infty & \infty & \infty \\
	\end{pmatrix}$
\end{center}
\vspace{5mm}
\par
\\
\hspace{5mm}
Составим таблицу итераций:
\\
\begin{center}
    \begin{tabular}{|c|c|c|c|c|c|c|c|c|c|c|c|c|c|c|c|c|}
        \hline 
        $ $ & $V_1$ & $V_2$ & $V_3$ & $V_4$ & $V_5$ & $V_6$ & $V_7$ & $V_8$ & $\lambda_i^{(0)}$ & $\lambda_i^{(1)}$ & $\lambda_i^{(2)}$ & $\lambda_i^{(3)}$ & $\lambda_i^{(4)}$ & $\lambda_i^{(5)}$ & $\lambda_i^{(6)}$ & $\lambda_i^{(7)}$ \\ \hline
        
        $V_1$ & 0 & 0 & 0 & 0 & 0 & 0 & 0 & 0 & \tikzmarknode{m1}{\Circled{0}} & 0 & 0 & 0 & 0 & 0 & 0 & 0 \\ \hline
        
        $V_2$ & $\infty$ & 3 & 3 & 3 & 3 & 3 & 3 & 3 & $\infty$ & \tikzmarknode{m2}{3} & 3 & 3 & 3 & 3 & 3 & 3 \\ \hline
        
        $V_3$ & $\infty$ & 5 & 4 & 4 & 4 & 4 & 4 & 4 & $\infty$ & 5 & \tikzmarknode{m3}{4} & 4 & 4 & 4 & 4 & 4 \\ \hline
        
        $V_4$ & $\infty$ & $\infty$ & 7 & 7 & 7 & 7 & 7 & 7 & $\infty$ & $\infty$ & \tikzmarknode{m4}{7} & 7 & 7 & 7 & 7 & 7 \\ \hline
        
        $V_5$ & $\infty$ & 6 & 6 & 6 & 6 & 6 & 6 & 6 & $\infty$ & \tikzmarknode{m5}{6} & 6 & 6 & 6 & 6 & 6 & 6 \\ \hline
        
        $V_6$ & $\infty$ & $\infty$ & 12 & 10 & 10 & 10 & 10 & 10 & $\infty$ & $\infty$ & 12 & \tikzmarknode{m6}{10} & 10 & 10 & 10 & 10 \\ \hline
        
        $V_7$ & $\infty$ & $\infty$ & $\infty$ & 12 & 12 & 12 & 12 & 12 & $\infty$ & $\infty$ & $\infty$ & \tikzmarknode{m7}{12} & 12 & 12 & 12 & 12 \\ \hline
        
        $V_8$ & $\infty$ & $\infty$ & 13 & 13 & 12 & 12 & 12 & 12 & $\infty$ & $\infty$ & 13 & 13 & \tikzmarknode{m8}{12} & 12 & 12 & 12 \\ \hline
    \end{tabular}
\end{center}

\begin{tikzpicture}[remember picture, overlay]
    \draw[->] (m1) -- (m2);
    \draw[->] (m1) -- (m5);
    \draw[->] (m2) -- (m3);
    \draw[->] (m2) -- (m4);
    \draw[->] (m4) -- (m6);
    \draw[->] (m4) -- (m7);
    \draw[->] (m6) -- (m8);
\end{tikzpicture}

\begin{spacing}{1.3}
\Circled{1} Минимальный путь из $v_1 \rightarrow v_2$: $v_1 \rightarrow v_2$ , его длина 3\\
\hspace*{11mm}
\lambda_1^{(0)}+C_{12} = 0 + 3 = \lambda_2^{(1)} \\

\Circled{2} Минимальный путь из $v_1 \rightarrow v_3$: $v_1 \rightarrow v_2$ \rightarrow v_3$, его длина 4\\
\hspace*{11mm}
\lambda_1^{(2)}+C_{23} = 3 + 1 = \lambda_3^{(2)} \\
\hspace*{12mm}
\lambda_1^{(0)}+C_{12} = 0 + 3 = \lambda_2^{(1)} \\

\Circled{3} Минимальный путь из $v_1 \rightarrow v_4$: $v_1 \rightarrow v_2 \rightarrow v_4$, его длина 5\\
\hspace*{11mm}
\lambda_2^{(1)}+C_{24} = 3 + 4 = \lambda_4^{(2)} \\
\hspace*{12mm}
\lambda_1^{(0)}+C_{12} = 0 + 3 = \lambda_2^{(1)} \\

\Circled{4} Минимальный путь из $v_1 \rightarrow v_5$: $v_1 \rightarrow v_5$, его длина 6\\
\hspace*{11mm}
\lambda_1^{(0)}+C_{15} = 0 + 6 = \lambda_5^{(1)} \\

\Circled{5} Минимальный путь из $v_1 \rightarrow v_6$: $v_1 \rightarrow v_2 \rightarrow v_4 \rightarrow v_6$, его длина 10\\
\hspace*{11mm}
\lambda_4^{(2)}+C_{46} = 7 + 3 = \lambda_6^{(3)} \\
\hspace*{12mm}
\lambda_2^{(1)}+C_{24} = 3 + 4 = \lambda_4^{(2)} \\
\hspace*{12mm}
\lambda_1^{(0)}+C_{12} = 0 + 3 = \lambda_2^{(1)} \\

\Circled{6} Минимальный путь из $v_1 \rightarrow v_7$: $v_1 \rightarrow v_2 \rightarrow v_4 \rightarrow v_7 $, его длина 12\\
\hspace*{11mm}
\lambda_4^{(2)}+C_{47} = 7 + 5 = \lambda_7^{(3)} \\
\hspace*{12mm}
\lambda_2^{(1)}+C_{24} = 3 + 4 = \lambda_4^{(2)} \\
\hspace*{12mm}
\lambda_1^{(0)}+C_{12} = 0 + 3 = \lambda_2^{(1)} \\


\Circled{7} Минимальный путь из $v_1 \rightarrow v_8$: $v_1 \rightarrow v_2 \rightarrow v_4 \rightarrow v_6 \rightarrow v_8$, его длина 12\\
\hspace*{11mm}
\lambda_6^{(3)}+C_{68} = 10 + 2 = \lambda_8^{(4)} \\
\hspace*{12mm}
\lambda_4^{(2)}+C_{46} = 7 + 3 = \lambda_6^{(3)} \\
\hspace*{12mm}
\lambda_2^{(1)}+C_{24} = 3 + 4 = \lambda_4^{(2)} \\
\hspace*{12mm}
\lambda_1^{(0)}+C_{12} = 0 + 3 = \lambda_2^{(1)} \\


\end{spacing}

\newpage
\begin{center}
	\textbf{Задание №5}
\end{center}
\begin{center}
	\begin{tikzpicture}
		[node distance={20mm}, thick, main/.style = {draw, circle}]
		\node[main] (1) {};
		\node[main] (2) [right of = 1] {};
		\node[main] (3) [right of = 2]{};
		\node[main] (4) [right of = 3]{};
		\node[main] (5) [below of = 1]{};
		\node[main] (6) [below of = 2]{};
		\node[main] (7) [below of = 3] {};
		\node[main] (8) [below of = 4] {};
		\node[main] (9) [below of = 5] {};
		\node[main] (10) [below of = 6] {};
		\node[main] (11) [below of = 7] {};
		\node[main] (12) [below of = 8] {};
		\draw (1) -- node[above] {2} (2);
		\draw (2) -- node[above] {6} (3);
		\draw (3) -- node[above] {7} (4);
		\draw (1) -- node[left] {1} (5);
		\draw (2) -- node[left] {5} (6);
		\draw (3) -- node[left] {5} (7);
		\draw (4) -- node[right] {6} (8);
		\draw (5) -- node[above] {8} (6);
		\draw (6) -- node[above] {4} (7);
		\draw (7) -- node[above] {1} (8);
		\draw (5) -- node[left] {7} (9);
		\draw (6) -- node[left] {5} (10);
		\draw (7) -- node[left] {5} (11);
		\draw (8) -- node[right] {5} (12);
		\draw (9) -- node[below] {9} (10);
		\draw (10) -- node[below] {7} (11);
		\draw (11) -- node[below] {3} (12);
	\end{tikzpicture} 
\end{center}\\
\hspace{6mm}Минимальный вес остовного дерева $L(D) = 44$\
Есть два варианта остовного дерева с минимальной суммой длин рёбер - $44:$
\begin{center}
	\begin{tikzpicture}
		[node distance={20mm}, thick, main/.style = {draw, circle}]
		\node[main] (1) {};
		\node[main] (2) [right of = 1] {};
		\node[main] (3) [right of = 2]{};
		\node[main] (4) [right of = 3]{};
		\node[main] (5) [below of = 1]{};
		\node[main] (6) [below of = 2]{};
		\node[main] (7) [below of = 3] {};
		\node[main] (8) [below of = 4] {};
		\node[main] (9) [below of = 5] {};
		\node[main] (10) [below of = 6] {};
		\node[main] (11) [below of = 7] {};
		\node[main] (12) [below of = 8] {};
		\draw (1) -- node[above] {2} (2);
		%\draw (2) -- node[above] {6} (3);
		%\draw (3) -- node[above] {7} (4);
		\draw (1) -- node[left] {1} (5);
		\draw (2) -- node[left] {5} (6);
		\draw (3) -- node[left] {5} (7);
		\draw (4) -- node[right] {6} (8);
		%\draw (5) -- node[above] {8} (6);
		\draw (6) -- node[above] {4} (7);
		\draw (7) -- node[above] {1} (8);
		\draw (5) -- node[left] {7} (9);
		\draw (6) -- node[left] {5} (10);
		\draw (7) -- node[left] {5} (11);
		%\draw (8) -- node[right] {5} (12);
		%\draw (9) -- node[below] {9} (10);
		%\draw (10) -- node[below] {7} (11);
		\draw (11) -- node[below] {3} (12);
	\end{tikzpicture} 
\end{center}\\
\begin{center}
	\begin{tikzpicture}
		[node distance={20mm}, thick, main/.style = {draw, circle}]
		\node[main] (1) {};
		\node[main] (2) [right of = 1] {};
		\node[main] (3) [right of = 2]{};
		\node[main] (4) [right of = 3]{};
		\node[main] (5) [below of = 1]{};
		\node[main] (6) [below of = 2]{};
		\node[main] (7) [below of = 3] {};
		\node[main] (8) [below of = 4] {};
		\node[main] (9) [below of = 5] {};
		\node[main] (10) [below of = 6] {};
		\node[main] (11) [below of = 7] {};
		\node[main] (12) [below of = 8] {};
		\draw (1) -- node[above] {2} (2);
		%\draw (2) -- node[above] {6} (3);
		%\draw (3) -- node[above] {7} (4);
		\draw (1) -- node[left] {1} (5);
		\draw (2) -- node[left] {5} (6);
		\draw (3) -- node[left] {5} (7);
		\draw (4) -- node[right] {6} (8);
		%\draw (5) -- node[above] {8} (6);
		\draw (6) -- node[above] {4} (7);
		\draw (7) -- node[above] {1} (8);
		\draw (5) -- node[left] {7} (9);
		\draw (6) -- node[left] {5} (10);
		%\draw (7) -- node[left] {5} (11);
		\draw (8) -- node[right] {5} (12);
		%\draw (9) -- node[below] {9} (10);
		%\draw (10) -- node[below] {7} (11);
		\draw (11) -- node[below] {3} (12);
	\end{tikzpicture} 
\end{center}\\
\newpage
\begin{center}
	\textbf{Задание №6}\\
\end{center}
\begin{center}
    \begin{tikzpicture}
        [node distance={30mm}, thick, main/.style = {draw, circle}]
        \node[main] (1) at (-3, 0) {};
        \node[main] (2) at (1, 2) {};
        \node[main] (3) at (5, 0) {};
        \node[main] (4) at (4, -3) {};
        \node[main] (5) at (0, -3) {};
        
        \draw (1) -- node[above left] {1} (2);
        \draw (1) -- node[above] {6} (3);
        \draw (1) -- node[above right] {7} (4);
        \draw (1) -- node[left] {5} (5);
        \draw (2) -- node[above right] {2} (3);
        \draw (2) -- node[right] {8} (4);
        \draw (3) -- node[right] {3} (4);
        \draw (3) -- node[below right] {9} (5);
        \draw (4) -- node[below] {4} (5);
    \end{tikzpicture}
\end{center}

\hspace*{8mm}1. Зададим произвольную ориентацию\\
\begin{center}
    \begin{tikzpicture}
        [node distance={30mm}, thick, main/.style = {draw, circle}]
        \node[main] (1) at (-3, 0) {};
        \node[main] (2) at (1, 2) {};
        \node[main] (3) at (5, 0) {};
        \node[main] (4) at (4, -3) {};
        \node[main] (5) at (0, -3) {};
        
        \draw [->](1) -- node[above left] {1} (2);
        \draw [->](1) -- node[above] {6} (3);
        \draw [<-](1) -- node[above right] {7} (4);
        \draw [<-](1) -- node[left] {5} (5);
        \draw [<-](2) -- node[above right] {2} (3);
        \draw [->](2) -- node[right] {8} (4);
        \draw [<-](3) -- node[right] {3} (4);
        \draw [->](3) -- node[below right] {9} (5);
        \draw [<-](4) -- node[below] {4} (5);
    \end{tikzpicture}
\end{center}

\noindent
\hspace*{8mm}2. Построим произвольное остовное дерево D\\
\begin{center}
    \begin{tikzpicture}
        [node distance={30mm}, thick, main/.style = {draw, circle}]
        \node[main] (1) at (-3, 0) {};
        \node[main] (2) at (1, 2) {};
        \node[main] (3) at (5, 0) {};
        \node[main] (4) at (4, -3) {};
        \node[main] (5) at (0, -3) {};
        
        \draw [->](1) -- node[above left] {q1} (2);
        %\draw [->](1) -- node[above] {q6} (3);
        %\draw [<-](1) -- node[above right] {q7} (4);
        %\draw [<-](1) -- node[left] {q5} (5);
        \draw [<-](2) -- node[above right] {q2} (3);
        %\draw [->](2) -- node[right] {q8} (4);
        \draw [<-](3) -- node[right] {q3} (4);
        %\draw [->](3) -- node[below right] {q9} (5);
        \draw [<-](4) -- node[below] {q4} (5);
    \end{tikzpicture}
\end{center}
\noindent

\newpage

\hspace*{11mm}
$(D + q_5): \mu_1 : V_1 - V_2 - V_3 - V_4 - V_5 - V_1$ \\
\hspace*{17mm}
$(D + q_7): \mu_2 : V_1 - V_2 - V_3 - V_4 - V_1$ \\
\hspace*{17mm}
$(D + q_9): \mu_3 : V_3 - V_4 - V_5 - V_3$ \\
\hspace*{17mm}
$(D + q_8): \mu_4 : V_2 - V_3 - V_4 - V_2$ \\
\hspace*{17mm}
$(D + q_6): \mu_5 : V_1 - V_2 - V_3 - V_1$

3. Составим цикломатическую матрицу:\\
\begin{center}
		\begin{tabular}{|c|c|c|c|c|c|c|c|c|c|c|c|c|}	
            \hline 
			$ \cellcolor{myblack}  $& $q_1$ & $q_2$ & $q_3$ & $q_4$ & $q_5$ & $q_6$ & $q_7$ & $q_8$ & $q_9$ \\ \hline
			$\mu_1$ & 1 & -1 & -1 & -1 & 1 & 0 & 0 & 0 & 0\\ \hline
			$\mu_2$ & 1 & -1 & 0 & 0 & 0 & -1 & 0 & 0 & 0 \\ \hline
			$\mu_3$ & 1 & -1 & -1 & 1 & 0 & 0 & 0 & 0 & 0\\ \hline
			$\mu_4$ & 0 & -1 & -1 & 0 & 0 & 0 & 0 & -1 & 0\\ \hline
			$\mu_5$ & 0 & 0 & -1 & -1 & 0 & 0 & 0 & 0 & -1\\ \hline

	   \end{tabular}
\end{center}
\vspace{5mm}\\ 
4. Запишем закон Кирхгофа для напряжений:
\vspace{5mm}
\\
$
\begin{cases}
u_1 - u_2 - u_3 - u_4 + u_5 = 0\\
u_1 - u_2 - u_6 = 0\\
u_1 - u_2 - u_3 + u_4 = 0\\
-u_2 - u_3 - u_8 = 0\\
-u_3 - u_4 + u_9 = 0\\

\end{cases}$\\
\vspace{5mm}
\begin{center}
    Напряжения, соответствующие ребрам, не вошедшим в остовное дерево – 
    базисные переменные системы.\\
\end{center}
\noindent
\hspace*{3mm}\\
5. Найдём матрицу инцидентности:
\hspace*{3mm}\\
\begin{center}
\begin{tabular}{|c|c|c|c|c|c|c|c|c|c|c|c|с|}
    \hline
    $ \cellcolor{myblack}  $ & \widetilde{$q_1$} & \widetilde{$q_2$} & \widetilde{$q_3$} & \widetilde{$q_4$} & \widetilde{$q_5$}& \widetilde{$q_6$} & \widetilde{$q_7$} & \widetilde{$q_8$} & \widetilde{$q_9$}\\
    \hline
    $\mu_1$ & -1 & 0 & 0 & 0 & 1 & -1 & 1 & 0 & 0\\ \hline
    $\mu_2$ & 1 & 1 & 0 & 0 & 0 & 0 & 0 & -1 & 0 \\ \hline
	$\mu_3$ & 0 & -1 & 1 & 0 & 0 & 1 & 0 & 0 & -1\\ \hline
	$\mu_4$ & 0 & 0 & -1 & 1 & 0 & 0 & -1 & 1 & 0\\ \hline
	$\mu_5$ & 0 & 0 & 0 & -1 & -1 & 0 & 0 & 0 & 1\\ \hline
\end{tabular}
\end{center}
\vspace{5mm}
\\

\definecolor{duskyPink}{rgb}{1, 0, 0} 
\begin{cases}
	I_1 + I_2 - I_8 = 0 \\ 
	-I_2 + I_3 + I_6 - I_9 = 0 \\
	-I_3 + I_4 - I_7 + I_8 = 0 \\
	-I_4 - I_5 + I_9 = 0 \\

\end{cases}
\vspace{5mm}\\
\noindent
6. Подставим закон Ома:
\vspace{5mm}
\\
\begin{cases}
\varepsilon_2 = -I_1R_1 + I_2R_2 + I_3R_3 + I_4R_4 \\
\varepsilon_1 = I_2R_2 - I_6R_6 \\
\varepsilon_1 = -I_3R_3 - I_2R_2 - I_4R_4 \\
0 = -I_2R_2 - I_3R_3 - I_8R_8\\
0 = -I_3R_3 - I_4R_4 - I_9R_9\\
\end{cases}\\
\\
7. Совместная система имеет вид:\\
\\
\begin{cases}
I_1 + I_2 - I_8 = 0 \\ 
-I_2 + I_3 + I_6 - I_9 = 0 \\
-I_3 + I_4 - I_7 + I_8 = 0 \\
\varepsilon_2 = -I_1R_1 + I_2R_2 + I_3R_3 + I_4R_4 \\
\varepsilon_1 = I_2R_2 - I_6R_6 \\
\varepsilon_1 = -I_3R_3 - I_2R_2 - I_4R_4 \\
0 = -I_2R_2 - I_3R_3 - I_8R_8\\
0 = -I_3R_3 - I_4R_4 - I_9R_9\\
\end{cases}


	\newpage
	\begin{center}
		\textbf{Задание №7}
	\end{center}
	\vspace{5mm}
\begin{tikzpicture}
	[node distance={20mm}, thick, main/.style = {draw, circle}]
	\hspace*{30mm}
	\node[main] (1) {$1$};
	\node[main] (2) [above right of = 1, above = 20pt, right = 3pt] {$2$};
	\node[main] (3) [above right of = 2, above = -5mm, right = 40pt, node distance={20mm}]{$3$};
	\node[main] (4) [below right of = 3, below = -5mm, right = 40pt]{$7$};
	\node[main] (5) [right of = 1, node distance={45mm}]{$5$};
	\node[main] (6) [right of = 5, node distance={65mm}]{$9$};
	\node[main] (7) [below right of = 1, below = 20pt, right = 3pt] {$4$};
	\node[main] (8) [below right of = 7, below = -5mm, right = 40pt, node distance={20mm}]{$6$};
	\node[main] (9) [above right of = 8, above= -5mm, right = 40pt]{$8$};
	\draw[->] (1) -- node[above] {3} (2);
	\draw[->] (2) -- node[above] {3} (3);
	\draw[->] (3) -- node[above] {9} (4);
	\draw[->] (4) -- node[above] {11} (6);
	\draw[->] (1) to [out=0,in=230,looseness=0.5] node[above = -2mm, left = 5mm] {10} (3);
	\draw[->] (1) -- node[above = 3mm, right=5mm] {5} (5);
	\draw[->] (5) -- node[above] {4} (6);
	\draw[->] (1) -- node[below] {3} (7);
	\draw[->] (7) -- node[below] {3} (8);
	\draw[->] (8) -- node[below] {7} (9);
	\draw[->] (9) -- node[below] {13} (6);
	\draw[->] (1) to [out=0,in=130,looseness=0.5] node[below = -2mm, left = 5mm] {7} (8);
	\draw[->] (2) -- node[above = 6mm, left = 1mm] {2} (5);
	\draw[->] (5) -- node[above] {3} (4);
	\draw[->] (7) -- node[above] {4} (5);
	\draw[->] (5) -- node[above] {5} (9);
\end{tikzpicture}
\vspace{5mm}
\par
Построение полного потока:\\
\begin{tikzpicture}
	[node distance={20mm}, thick, main/.style = {draw, circle}]
	\hspace*{30mm}
	\node[main] (1) {$1$};
	\node[main] (2) [above right of = 1, above = 20pt, right = 3pt] {$2$};
	\node[main] (3) [above right of = 2, above = -5mm, right = 40pt, node distance={20mm}]{$3$};
	\node[main] (4) [below right of = 3, below = -5mm, right = 40pt]{$7$};
	\node[main] (5) [right of = 1, node distance={45mm}]{$5$};
	\node[main] (6) [right of = 5, node distance={65mm}]{$9$};
	\node[main] (7) [below right of = 1, below = 20pt, right = 3pt] {$4$};
	\node[main] (8) [below right of = 7, below = -5mm, right = 40pt, node distance={20mm}]{$6$};
	\node[main] (9) [above right of = 8, above= -5mm, right = 40pt]{$8$};
	\draw[->] (1) -- node[above] {3} (2);
	\draw[->] (2) -- node[above] {3} (3);
	\draw[->] (3) -- node[above] {9} (4);
	\draw[->] (4) -- node[above] {10} (6);
	\draw[->] (1) to [out=0,in=230,looseness=0.5] node[above = -2mm, left = 5mm] {6} (3);
	\draw[->] (1) -- node[above = 3mm, right=5mm] {5} (5);
	\draw[->] (5) -- node[above] {4} (6);
	\draw[->] (1) -- node[below] {3} (7);
	\draw[->] (7) -- node[below] {3} (8);
	\draw[->] (8) -- node[below] {7} (9);
	\draw[->] (9) -- node[below] {7} (6);
	\draw[->] (1) to [out=0,in=130,looseness=0.5] node[below = -2mm, left = 5mm] {4} (8);
	\draw[->] (2) -- node[above = 6mm, left = 1mm] {2} (5);
	\draw[->] (5) -- node[above] {1} (4);
	\draw[->] (7) -- node[above] {0} (5);
	\draw[->] (5) -- node[above] {0} (9);
\end{tikzpicture}

\begin{enumerate} 
	\setlength{\itemindent}{3mm}
	\item $v_1 - v_2 - v_3 - v_7 - v_9$
	\begin{itemize}
		\setlength{\itemindent}{3mm}
		\item $\min\{3-0, 3-0, 9-0, 11-0\} = 3$
	\end{itemize}
	\item $v_1 - v_4 - v_6 - v_8 - v_9$
	\begin{itemize}
		\setlength{\itemindent}{3mm}
		\item $\min\{3-0, 3-0, 7-0, 11-0\} = 3$
	\end{itemize}
	\item $v_1 - v_5 - v_9$
	\begin{itemize}
		\setlength{\itemindent}{3mm}
		\item $\min\{5-0, 4-0\} = 4$
	\end{itemize}
	\item $v_1 - v_3 - v_7 - v_9$
	\begin{itemize}
		\setlength{\itemindent}{3mm}
		\item $\min\{10-0, 9-3, 11-3\} = 6$
	\end{itemize}
\newpage
	\item $v_1 - v_6 - v_8 - v_9$
	\begin{itemize}
		\setlength{\itemindent}{3mm}
		\item $\min\{7-0, 7-3, 13-3\} = 4$
	\end{itemize}
	\item $v_1 - v_5 - v_7 - v_9$
	\begin{itemize}
		\setlength{\itemindent}{3mm}
		\item $\min\{5-4, 3-0, 11-9\} = 1$
	\end{itemize}
\end{enumerate}
\par
Величина полного потока $\Phi_{\text{пол.}} = 3 + 6 + 5 + 4 + 3 = 21 = 10 + 4 + 7$\\\\\
\par
Построение максимального потока\\\\
\vspace{5mm}
\begin{tikzpicture}
	[node distance={20mm}, thick, main/.style = {draw, circle}]
	\hspace*{30mm}
	\node[main] (1) {$1$};
	\node[main] (2) [above right of = 1, above = 20pt, right = 3pt] {$2$};
	\node[main] (3) [above right of = 2, above = -5mm, right = 40pt, node distance={20mm}]{$3$};
	\node[main] (4) [below right of = 3, below = -5mm, right = 40pt]{$7$};
	\node[main] (5) [right of = 1, node distance={45mm}]{$5$};
	\node[main] (6) [right of = 5, node distance={65mm}]{$9$};
	\node[main] (7) [below right of = 1, below = 20pt, right = 3pt] {$4$};
	\node[main] (8) [below right of = 7, below = -5mm, right = 40pt, node distance={20mm}]{$6$};
	\node[main] (9) [above right of = 8, above= -5mm, right = 40pt]{$8$};
	\draw[->] (1) -- node[above] {3} (2);
	\draw[->] (2) -- node[above] {2} (3);
	\draw[->] (3) -- node[above] {9} (4);
	\draw[->] (4) -- node[above] {10} (6);
	\draw[->] (1) to [out=0,in=230,looseness=0.5] node[above = -2mm, left = 5mm] {7} (3);
	\draw[->] (1) -- node[above = 3mm, right=5mm] {5} (5);
	\draw[->] (5) -- node[above] {4} (6);
	\draw[->] (1) -- node[below] {3} (7);
	\draw[->] (7) -- node[below] {0} (8);
	\draw[->] (8) -- node[below] {7} (9);
	\draw[->] (9) -- node[below] {10} (6);
	\draw[->] (1) to [out=0,in=130,looseness=0.5] node[below = -2mm, left = 5mm] {7} (8);
	\draw[->] (2) -- node[above = 6mm, left = 1mm] {1} (5);
	\draw[->] (5) -- node[above] {2} (4);
	\draw[->] (7) -- node[above] {3} (5);
	\draw[->] (5) -- node[above] {3} (9);
\end{tikzpicture}

\par
\begin{enumerate} 
	\setlength{\itemindent}{3mm}
	\item $v_1 - v_3 - v_2 - v_5 - v_7 - v_9$
	\begin{itemize}
		\setlength{\itemindent}{3mm}
		\item $\Delta_1 = \min\{10-6, 3, 2-0, 3-1, 11-10\} = 1$
	\end{itemize}
	\item $v_1 - v_6 - v_4 - v_5 - v_8 - v_9$
	\begin{itemize}
		\setlength{\itemindent}{3mm}
		\item $\Delta_2 = \min\{7-4, 3, 6-0, 5-0 13-7\} = 3$
        \end{itemize}
        \item $v_1 - v_3 - v_2 - v_5 - v_8 - v_9$
	\begin{itemize}
		\setlength{\itemindent}{3mm}
	    \item $\Delta_3 = \min\{10-7, 3-2, 2-1, 5-3, 13-10\} = 1$
	\end{itemize}
\end{enumerate}
\par
Величина максимального потока $\Phi_{\text{макс.}} =11 + 1 + 10 + 4 = 26 = 3 + 7 + 5 + 7 + 3 + 1$	\\

\end{document}